\documentclass{sig-alternate}
\usepackage[utf8]{inputenc}

\begin{document}

\conferenceinfo{MSR}{Mining Software Repositories '14}

\title{How to Calculate Code Coverage\\
Without Running the Tests?\\
An Heuristic Using Statical Code Analysis}

\numberofauthors{2}

\author{
\alignauthor
Mauricio Aniche\\
\affaddr{Institute of Mathematics and Statistics}\\
\affaddr{University of São Paulo}\\
\email{aniche@ime.usp.br}
\alignauthor
Marco Aurelio Gerosa\\
\affaddr{Institute of Mathematics and Statistics}\\
\affaddr{University of São Paulo}\\
\email{gerosa@ime.usp.br}
}

\maketitle
\begin{abstract}

bla bla bla

\end{abstract}

\section{Introduction}

falar um pouco de msr

problemas de se compilar codigo

code coverage precisa executar

solucao estatica tem erros, mas pode ser aceitavel

\cite{bowman:reasoning}

\section{Code Coverage}

falar um pouco disso

\section{Experiment Design}

explica como fizemos

instrumentamos um codigo com aspectos, parseamos, calculamos

epxlica a formula

\section{Results}

mostra os graficos bla bla

\section{Discussion}

\section{Related Work}

\section{Acknowledgments}

bla bla

\bibliographystyle{abbrv}
\bibliography{sigproc}

\end{document}
